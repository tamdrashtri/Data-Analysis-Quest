\documentclass[]{article}
\usepackage{lmodern}
\usepackage{amssymb,amsmath}
\usepackage{ifxetex,ifluatex}
\usepackage{fixltx2e} % provides \textsubscript
\ifnum 0\ifxetex 1\fi\ifluatex 1\fi=0 % if pdftex
  \usepackage[T1]{fontenc}
  \usepackage[utf8]{inputenc}
\else % if luatex or xelatex
  \ifxetex
    \usepackage{mathspec}
  \else
    \usepackage{fontspec}
  \fi
  \defaultfontfeatures{Ligatures=TeX,Scale=MatchLowercase}
\fi
% use upquote if available, for straight quotes in verbatim environments
\IfFileExists{upquote.sty}{\usepackage{upquote}}{}
% use microtype if available
\IfFileExists{microtype.sty}{%
\usepackage{microtype}
\UseMicrotypeSet[protrusion]{basicmath} % disable protrusion for tt fonts
}{}
\usepackage[margin=1in]{geometry}
\usepackage{hyperref}
\hypersetup{unicode=true,
            pdftitle={Assignment/Homework 1 - Tam Nguyen},
            pdfborder={0 0 0},
            breaklinks=true}
\urlstyle{same}  % don't use monospace font for urls
\usepackage{color}
\usepackage{fancyvrb}
\newcommand{\VerbBar}{|}
\newcommand{\VERB}{\Verb[commandchars=\\\{\}]}
\DefineVerbatimEnvironment{Highlighting}{Verbatim}{commandchars=\\\{\}}
% Add ',fontsize=\small' for more characters per line
\usepackage{framed}
\definecolor{shadecolor}{RGB}{248,248,248}
\newenvironment{Shaded}{\begin{snugshade}}{\end{snugshade}}
\newcommand{\KeywordTok}[1]{\textcolor[rgb]{0.13,0.29,0.53}{\textbf{#1}}}
\newcommand{\DataTypeTok}[1]{\textcolor[rgb]{0.13,0.29,0.53}{#1}}
\newcommand{\DecValTok}[1]{\textcolor[rgb]{0.00,0.00,0.81}{#1}}
\newcommand{\BaseNTok}[1]{\textcolor[rgb]{0.00,0.00,0.81}{#1}}
\newcommand{\FloatTok}[1]{\textcolor[rgb]{0.00,0.00,0.81}{#1}}
\newcommand{\ConstantTok}[1]{\textcolor[rgb]{0.00,0.00,0.00}{#1}}
\newcommand{\CharTok}[1]{\textcolor[rgb]{0.31,0.60,0.02}{#1}}
\newcommand{\SpecialCharTok}[1]{\textcolor[rgb]{0.00,0.00,0.00}{#1}}
\newcommand{\StringTok}[1]{\textcolor[rgb]{0.31,0.60,0.02}{#1}}
\newcommand{\VerbatimStringTok}[1]{\textcolor[rgb]{0.31,0.60,0.02}{#1}}
\newcommand{\SpecialStringTok}[1]{\textcolor[rgb]{0.31,0.60,0.02}{#1}}
\newcommand{\ImportTok}[1]{#1}
\newcommand{\CommentTok}[1]{\textcolor[rgb]{0.56,0.35,0.01}{\textit{#1}}}
\newcommand{\DocumentationTok}[1]{\textcolor[rgb]{0.56,0.35,0.01}{\textbf{\textit{#1}}}}
\newcommand{\AnnotationTok}[1]{\textcolor[rgb]{0.56,0.35,0.01}{\textbf{\textit{#1}}}}
\newcommand{\CommentVarTok}[1]{\textcolor[rgb]{0.56,0.35,0.01}{\textbf{\textit{#1}}}}
\newcommand{\OtherTok}[1]{\textcolor[rgb]{0.56,0.35,0.01}{#1}}
\newcommand{\FunctionTok}[1]{\textcolor[rgb]{0.00,0.00,0.00}{#1}}
\newcommand{\VariableTok}[1]{\textcolor[rgb]{0.00,0.00,0.00}{#1}}
\newcommand{\ControlFlowTok}[1]{\textcolor[rgb]{0.13,0.29,0.53}{\textbf{#1}}}
\newcommand{\OperatorTok}[1]{\textcolor[rgb]{0.81,0.36,0.00}{\textbf{#1}}}
\newcommand{\BuiltInTok}[1]{#1}
\newcommand{\ExtensionTok}[1]{#1}
\newcommand{\PreprocessorTok}[1]{\textcolor[rgb]{0.56,0.35,0.01}{\textit{#1}}}
\newcommand{\AttributeTok}[1]{\textcolor[rgb]{0.77,0.63,0.00}{#1}}
\newcommand{\RegionMarkerTok}[1]{#1}
\newcommand{\InformationTok}[1]{\textcolor[rgb]{0.56,0.35,0.01}{\textbf{\textit{#1}}}}
\newcommand{\WarningTok}[1]{\textcolor[rgb]{0.56,0.35,0.01}{\textbf{\textit{#1}}}}
\newcommand{\AlertTok}[1]{\textcolor[rgb]{0.94,0.16,0.16}{#1}}
\newcommand{\ErrorTok}[1]{\textcolor[rgb]{0.64,0.00,0.00}{\textbf{#1}}}
\newcommand{\NormalTok}[1]{#1}
\usepackage{graphicx,grffile}
\makeatletter
\def\maxwidth{\ifdim\Gin@nat@width>\linewidth\linewidth\else\Gin@nat@width\fi}
\def\maxheight{\ifdim\Gin@nat@height>\textheight\textheight\else\Gin@nat@height\fi}
\makeatother
% Scale images if necessary, so that they will not overflow the page
% margins by default, and it is still possible to overwrite the defaults
% using explicit options in \includegraphics[width, height, ...]{}
\setkeys{Gin}{width=\maxwidth,height=\maxheight,keepaspectratio}
\IfFileExists{parskip.sty}{%
\usepackage{parskip}
}{% else
\setlength{\parindent}{0pt}
\setlength{\parskip}{6pt plus 2pt minus 1pt}
}
\setlength{\emergencystretch}{3em}  % prevent overfull lines
\providecommand{\tightlist}{%
  \setlength{\itemsep}{0pt}\setlength{\parskip}{0pt}}
\setcounter{secnumdepth}{0}
% Redefines (sub)paragraphs to behave more like sections
\ifx\paragraph\undefined\else
\let\oldparagraph\paragraph
\renewcommand{\paragraph}[1]{\oldparagraph{#1}\mbox{}}
\fi
\ifx\subparagraph\undefined\else
\let\oldsubparagraph\subparagraph
\renewcommand{\subparagraph}[1]{\oldsubparagraph{#1}\mbox{}}
\fi

%%% Use protect on footnotes to avoid problems with footnotes in titles
\let\rmarkdownfootnote\footnote%
\def\footnote{\protect\rmarkdownfootnote}

%%% Change title format to be more compact
\usepackage{titling}

% Create subtitle command for use in maketitle
\newcommand{\subtitle}[1]{
  \posttitle{
    \begin{center}\large#1\end{center}
    }
}

\setlength{\droptitle}{-2em}

  \title{Assignment/Homework 1 - Tam Nguyen}
    \pretitle{\vspace{\droptitle}\centering\huge}
  \posttitle{\par}
    \author{}
    \preauthor{}\postauthor{}
    \date{}
    \predate{}\postdate{}
  

\begin{document}
\maketitle

\paragraph{1.}\label{section}

Identify whether the following variables are numerical or categorical.
If numerical, state whether the variable is discrete or continuous. If
categorical, state whether the categories have a natural order (ordinal)
or not (nominal).

\begin{enumerate}
\def\labelenumi{\alph{enumi}.}
\tightlist
\item
  Number of sexual partners in a year
\end{enumerate}

\subparagraph{numerical, discrete}\label{numerical-discrete}

\begin{enumerate}
\def\labelenumi{\alph{enumi}.}
\setcounter{enumi}{1}
\tightlist
\item
  Petal area of rose flowers
\end{enumerate}

\subparagraph{numerical, continuous}\label{numerical-continuous}

\begin{enumerate}
\def\labelenumi{\alph{enumi}.}
\setcounter{enumi}{2}
\tightlist
\item
  Heart beats per minute of a Tour de France cyclist, averaged over the
  duration of the race
\end{enumerate}

\subparagraph{numerical, discrete}\label{numerical-discrete-1}

\begin{enumerate}
\def\labelenumi{\alph{enumi}.}
\setcounter{enumi}{3}
\tightlist
\item
  Birth weight
\end{enumerate}

\subparagraph{numerical, continuous}\label{numerical-continuous-1}

\begin{enumerate}
\def\labelenumi{\alph{enumi}.}
\setcounter{enumi}{4}
\tightlist
\item
  Stage of fruit ripeness (e.g., underripe, ripe, or overripe)
\end{enumerate}

\subparagraph{categorical, nominal}\label{categorical-nominal}

\begin{enumerate}
\def\labelenumi{\alph{enumi}.}
\setcounter{enumi}{5}
\tightlist
\item
  Angle of flower orientation relative to position of the sun
\end{enumerate}

\subparagraph{numerical , continuous}\label{numerical-continuous-2}

\begin{enumerate}
\def\labelenumi{\alph{enumi}.}
\setcounter{enumi}{6}
\tightlist
\item
  Tree species
\end{enumerate}

\subparagraph{categorical, nominal}\label{categorical-nominal-1}

\begin{enumerate}
\def\labelenumi{\alph{enumi}.}
\setcounter{enumi}{7}
\tightlist
\item
  Year of birth
\end{enumerate}

\subparagraph{numerical, discrete}\label{numerical-discrete-2}

\begin{enumerate}
\def\labelenumi{\roman{enumi}.}
\tightlist
\item
  Gender
\end{enumerate}

\subparagraph{categorical, nominal}\label{categorical-nominal-2}

\paragraph{2.}\label{section-1}

Not all telephone polls carried out to estimate voter or consumer
preferences make calls to cell phones. One reason is that in the USA
automated calls (``robocalls'') to cell phones are not permitted, and
interviews conducted by humans are more costly.

\begin{enumerate}
\def\labelenumi{\alph{enumi}.}
\tightlist
\item
  How might the strategy of leaving out cell phones affect the goal of
  obtaining a random sample of voters or consumers?
\end{enumerate}

\subparagraph{this might lead to bias. They are members that are
difficult to collect, but leaving them out will lead to an
under-representaion of the
population}\label{this-might-lead-to-bias.-they-are-members-that-are-difficult-to-collect-but-leaving-them-out-will-lead-to-an-under-representaion-of-the-population}

\begin{enumerate}
\def\labelenumi{\alph{enumi}.}
\setcounter{enumi}{1}
\tightlist
\item
  Which criterion of random sampling is most likely to be violated by
  the problems you identified in part (a): equal chance of being
  selected, or the independence of the selection of individuals?
\end{enumerate}

\subparagraph{equal chance of being selected - these difficult to
collect individuals might have characteristics differed from the rest of
the population, so they can be quite
important.}\label{equal-chance-of-being-selected---these-difficult-to-collect-individuals-might-have-characteristics-differed-from-the-rest-of-the-population-so-they-can-be-quite-important.}

\begin{enumerate}
\def\labelenumi{\alph{enumi}.}
\setcounter{enumi}{2}
\tightlist
\item
  Which attribute of estimated consumer preference is most affected by
  the problem you identified in (a): accuracy or precision?
\end{enumerate}

\subparagraph{this is related to accuracy - the result of the sample not
properly
taken.}\label{this-is-related-to-accuracy---the-result-of-the-sample-not-properly-taken.}

\paragraph{3.}\label{section-2}

In each of the following examples, indicate which variable is the
explanatory variable and which is the response variable.

\begin{enumerate}
\def\labelenumi{\alph{enumi}.}
\tightlist
\item
  The anticoagulant warfarin is often used as a pesticide against house
  mice, Mus musculus. Some populations of the house mouse have acquired
  a mutation in the vkorc1 gene from hybridizing with the Algerian
  mouse, M. spretus (Song et al. 2011). In the Algerian mice, this gene
  confers resistance to warfarin. In a hypothetical follow-up study,
  researchers collected a sample of house mice to determine whether
  presence of the vkorc1 mutation is associated with warfarin resistance
  in house mice as well. They fed warfarin to all the mice in a sample
  and compared survival between the individuals possessing the mutation
  and those not possessing the mutation.
\end{enumerate}

\subparagraph{Response variable: survival
rate}\label{response-variable-survival-rate}

\subparagraph{Explanatory variable: the vkorc1
mutation}\label{explanatory-variable-the-vkorc1-mutation}

\begin{enumerate}
\def\labelenumi{\alph{enumi}.}
\setcounter{enumi}{1}
\tightlist
\item
  Cooley et al. (2009) randomly assigned either of two treatments,
  naturopathic care (diet counseling, breathing techniques, vitamins and
  a herbal medicine) or standardized psychotherapy (psychotherapy with
  breathing techniques and a placebo added), to 81 individuals having
  moderate to severe anxiety. Anxiety scores decreased an average of
  57\% in the naturopathic group and 31\% in the psychotherapy group.
\end{enumerate}

\subparagraph{Explanatory variable: naturopathic care, standardized
psychotherapy}\label{explanatory-variable-naturopathic-care-standardized-psychotherapy}

\subparagraph{Response variable: anxiety
level}\label{response-variable-anxiety-level}

\begin{enumerate}
\def\labelenumi{\alph{enumi}.}
\setcounter{enumi}{2}
\tightlist
\item
  Individuals highly sensitive to rewards tend to experience more food
  cravings and are more likely to be overweight or develop eating
  disorders than other people. Beaver et al. (2006) recruited 14 healthy
  volunteers and scored their reward sensitivity using a questionnaire
  (they were asked to answer ``yes'' or ``no'' to questions like: ``I'm
  always willing to try something new if I think it will be fun''). The
  subjects were then presented with images of appetizing foods (e.g.,
  chocolate cake, pizza) while activity of their
  fronto--striatal--amygdala--midbrain was measured using functional
  MRI. Reward sensitivity of subjects was found to correlate with brain
  activity in response to the images.
\end{enumerate}

\subparagraph{Explanatory variable: activity of participants'
fronto--striatal--amygdala--midbrain}\label{explanatory-variable-activity-of-participants-frontostriatalamygdalamidbrain}

\subparagraph{Response variable: reward
sensitivity}\label{response-variable-reward-sensitivity}

\begin{enumerate}
\def\labelenumi{\alph{enumi}.}
\setcounter{enumi}{3}
\tightlist
\item
  Endostatin is a naturally occurring protein in humans and mice that
  inhibits the growth of blood vessels. O'Reilly et al. (1997)
  investigated its effects on growth of cancer tumors, whose growth and
  spread requires blood vessel proliferation. Mice having lung carcinoma
  tumors were randomly divided into groups that were treated with doses
  of either 0, 2.5, 10, and 20 mg/kg of endostatin injected once daily.
  They found that higher doses of endostatin led to inhibition of tumor
  growth.
\end{enumerate}

\subparagraph{Explanatory variable: doses of
edostatin}\label{explanatory-variable-doses-of-edostatin}

\subparagraph{Response variable: tumor
growth}\label{response-variable-tumor-growth}

\paragraph{4.}\label{section-3}

For each of the studies presented in problem 3, indicate whether the
study is an experimental or observational study.

\subparagraph{1. house mice study: experimental
study}\label{house-mice-study-experimental-study}

\subparagraph{2. anxiety study: experimental
study}\label{anxiety-study-experimental-study}

\subparagraph{3. food craving study: observational
study}\label{food-craving-study-observational-study}

\subparagraph{4. tumor growth study: experimental
study}\label{tumor-growth-study-experimental-study}

\paragraph{5.}\label{section-4}

The Cambridge Study in Delinquent Development was undertaken in north
London (UK) to investigate the links between criminal behavior in young
men and the socioeconomic factors of their upbringing (Farrington 1994).
A cohort of 395 boys was followed for about 20 years, starting at the
age of eight or nine. All of the boys attended six schools located near
the research office. The following table shows the total number of
criminal convictions by the boys between the start and end of the study.

\begin{enumerate}
\def\labelenumi{\alph{enumi}.}
\tightlist
\item
  What type of table is this?
\end{enumerate}

\subparagraph{frequency table}\label{frequency-table}

\begin{enumerate}
\def\labelenumi{\alph{enumi}.}
\setcounter{enumi}{1}
\tightlist
\item
  How many variables are presented in this table?
\end{enumerate}

\subparagraph{there is only one numerical variable: number of
convictions}\label{there-is-only-one-numerical-variable-number-of-convictions}

\begin{enumerate}
\def\labelenumi{\alph{enumi}.}
\setcounter{enumi}{2}
\tightlist
\item
  How many boys had exactly two convictions by the end of the study?
\end{enumerate}

\subparagraph{21}\label{section-5}

\begin{enumerate}
\def\labelenumi{\alph{enumi}.}
\setcounter{enumi}{3}
\tightlist
\item
  What fraction of boys had no convictions?
\end{enumerate}

\subparagraph{0.67 or 67\% of boys have no
conviction}\label{or-67-of-boys-have-no-conviction}

\begin{enumerate}
\def\labelenumi{\alph{enumi}.}
\setcounter{enumi}{4}
\tightlist
\item
  Display the frequency distribution in a graph. Which type of graph is
  most appropriate? Why?
\end{enumerate}

\begin{Shaded}
\begin{Highlighting}[]
\NormalTok{graph1 <-}\StringTok{ }\KeywordTok{data.frame}\NormalTok{(}\StringTok{"Number of Convictions"}\NormalTok{ =}\StringTok{ }\DecValTok{0}\OperatorTok{:}\DecValTok{14}\NormalTok{, }\StringTok{"Frequency"}\NormalTok{ =}\StringTok{ }\KeywordTok{c}\NormalTok{(}\DecValTok{265}\NormalTok{, }\DecValTok{49}\NormalTok{, }\DecValTok{21}\NormalTok{, }\DecValTok{19}\NormalTok{, }\DecValTok{10}\NormalTok{, }\DecValTok{10}\NormalTok{, }\DecValTok{2}\NormalTok{, }\DecValTok{2}\NormalTok{, }\DecValTok{4}\NormalTok{, }\DecValTok{2}\NormalTok{, }\DecValTok{1}\NormalTok{, }\DecValTok{4}\NormalTok{, }\DecValTok{3}\NormalTok{, }\DecValTok{1}\NormalTok{, }\DecValTok{2}\NormalTok{))}

\KeywordTok{barplot}\NormalTok{(graph1}\OperatorTok{$}\NormalTok{Frequency, }
        \DataTypeTok{axisnames =} \OtherTok{TRUE}\NormalTok{,}
        \DataTypeTok{names.arg =}\NormalTok{ graph1}\OperatorTok{$}\NormalTok{Number.of.Convictions,}
        \DataTypeTok{xlab =} \StringTok{"Number of Convictions"}\NormalTok{,}
        \DataTypeTok{ylab =} \StringTok{"Frequency"}\NormalTok{)}
\end{Highlighting}
\end{Shaded}

\includegraphics{Assignment_1_files/figure-latex/unnamed-chunk-1-1.pdf}

\begin{Shaded}
\begin{Highlighting}[]
\KeywordTok{ggplot}\NormalTok{(graph1, }\KeywordTok{aes}\NormalTok{(Number.of.Convictions, Frequency)) }\OperatorTok{+}
\StringTok{  }\KeywordTok{geom_col}\NormalTok{()}
\end{Highlighting}
\end{Shaded}

\includegraphics{Assignment_1_files/figure-latex/unnamed-chunk-1-2.pdf}

\subparagraph{bar graph is the most appropiate since the number of
convictions is a numeric and dicreet variable and it helps to visualise
the frequency of each number of
convictions.}\label{bar-graph-is-the-most-appropiate-since-the-number-of-convictions-is-a-numeric-and-dicreet-variable-and-it-helps-to-visualise-the-frequency-of-each-number-of-convictions.}

\begin{enumerate}
\def\labelenumi{\alph{enumi}.}
\setcounter{enumi}{5}
\tightlist
\item
  Describe the shape of the frequency distribution. Is it skewed or is
  it symmetric? Is it unimodal or bimodal? Where is the mode in number
  of criminal convictions? Are there outliers in the number of
  convictions?
\end{enumerate}

\subparagraph{the shape is skewed and it is unimodal. The mode is 0 in
terms of the number of criminal convictions. 0 conviction is also an
outlier in the number of conviction because of it large frequency
compared to the
rest.}\label{the-shape-is-skewed-and-it-is-unimodal.-the-mode-is-0-in-terms-of-the-number-of-criminal-convictions.-0-conviction-is-also-an-outlier-in-the-number-of-conviction-because-of-it-large-frequency-compared-to-the-rest.}

\begin{enumerate}
\def\labelenumi{\alph{enumi}.}
\setcounter{enumi}{6}
\tightlist
\item
  Does the sample of boys used in this study represent a random sample
  of British boys? Why or why not?
\end{enumerate}

\subparagraph{This sample does not represent a random sample because it
is stated that all the boys are recruited in the study are all located
near the research office. Boys in other areas are not recruited. So this
lead to bias and affects the accuracy of the
sample.}\label{this-sample-does-not-represent-a-random-sample-because-it-is-stated-that-all-the-boys-are-recruited-in-the-study-are-all-located-near-the-research-office.-boys-in-other-areas-are-not-recruited.-so-this-lead-to-bias-and-affects-the-accuracy-of-the-sample.}

\paragraph{6.}\label{section-6}

The following graph was drawn using a very popular spreadsheet program
in an attempt to show the frequencies of observations in four
hypothetical groups. Before reading further, estimate by eye the
frequencies in each of the four groups.

\begin{enumerate}
\def\labelenumi{\alph{enumi}.}
\tightlist
\item
  Identify two features of this graph that cause it to violate the
  principle, ``Make patterns in the data easy to see.''
\end{enumerate}

\subparagraph{1. The angled perspective of the 3D graph makes it
difficult to judge the bar height by eye. It makes the pattern harder to
see.}\label{the-angled-perspective-of-the-3d-graph-makes-it-difficult-to-judge-the-bar-height-by-eye.-it-makes-the-pattern-harder-to-see.}

\subparagraph{2. The units in vertical axis cram too many numbers. It
adds a number in every 5 points. Better to show a number every 10
points, like 0, 10, 20, 30, 40 instead of 0, 5,
10\ldots{}}\label{the-units-in-vertical-axis-cram-too-many-numbers.-it-adds-a-number-in-every-5-points.-better-to-show-a-number-every-10-points-like-0-10-20-30-40-instead-of-0-5-10}

\begin{enumerate}
\def\labelenumi{\alph{enumi}.}
\setcounter{enumi}{1}
\tightlist
\item
  Identify at least two other features of the graph that make it
  difficult to interpret.
\end{enumerate}

\subparagraph{1. Graphical elements are not clearly labeled. The
vertical axis is not
labeled.}\label{graphical-elements-are-not-clearly-labeled.-the-vertical-axis-is-not-labeled.}

\subparagraph{2. the colors, shapes and shadows of this graph make it
overcomplicated while not communicating any further information. Colors
in this graph are also not differ in intensity and shapes, making it
hard for readers to distinguish between groups in the graph. The shape
of this graph is shaped like cones, making it harder to know where is
the exact point of frequency to line up with the vertical
axis.}\label{the-colors-shapes-and-shadows-of-this-graph-make-it-overcomplicated-while-not-communicating-any-further-information.-colors-in-this-graph-are-also-not-differ-in-intensity-and-shapes-making-it-hard-for-readers-to-distinguish-between-groups-in-the-graph.-the-shape-of-this-graph-is-shaped-like-cones-making-it-harder-to-know-where-is-the-exact-point-of-frequency-to-line-up-with-the-vertical-axis.}

\begin{enumerate}
\def\labelenumi{\alph{enumi}.}
\setcounter{enumi}{2}
\tightlist
\item
  The actual frequencies are 10, 20, 30, and 40. Draw a graph that
  overcomes the problems identified above.
\end{enumerate}

\begin{Shaded}
\begin{Highlighting}[]
\NormalTok{graph2 <-}\StringTok{ }\KeywordTok{data.frame}\NormalTok{(}\StringTok{"Group"}\NormalTok{ =}\StringTok{ }\KeywordTok{c}\NormalTok{(}\StringTok{"Group 1"}\NormalTok{, }\StringTok{"Group 2"}\NormalTok{, }\StringTok{"Group 3"}\NormalTok{, }\StringTok{"Group 4"}\NormalTok{), }\StringTok{"Frequency"}\NormalTok{ =}\StringTok{ }\KeywordTok{c}\NormalTok{(}\DecValTok{10}\NormalTok{, }\DecValTok{20}\NormalTok{, }\DecValTok{30}\NormalTok{, }\DecValTok{40}\NormalTok{))}


\KeywordTok{barplot}\NormalTok{(graph2}\OperatorTok{$}\NormalTok{Frequency,}
        \DataTypeTok{names.arg =}\NormalTok{ graph2}\OperatorTok{$}\NormalTok{Group,}
        \DataTypeTok{xlab =} \StringTok{"Group"}\NormalTok{,}
        \DataTypeTok{ylab =} \StringTok{"Frequency"}\NormalTok{)}
\end{Highlighting}
\end{Shaded}

\includegraphics{Assignment_1_files/figure-latex/unnamed-chunk-2-1.pdf}

\begin{Shaded}
\begin{Highlighting}[]
\KeywordTok{ggplot}\NormalTok{(graph2, }\KeywordTok{aes}\NormalTok{(Group, Frequency)) }\OperatorTok{+}
\StringTok{  }\KeywordTok{geom_col}\NormalTok{()}
\end{Highlighting}
\end{Shaded}

\includegraphics{Assignment_1_files/figure-latex/unnamed-chunk-2-2.pdf}


\end{document}
